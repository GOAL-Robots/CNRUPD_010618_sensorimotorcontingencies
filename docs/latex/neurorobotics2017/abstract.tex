The first `object' that newborn children start start to play with is their own body. This activity allows them to autonomously form a sensorimotor map and a repertoire of actions that constitutes the core of future cognitive and motor development. In this work we propose a computational model incorporating the hypothesis that this acquisition of early knowledge is not guided by random motor-babbling, but rather by goals autonomously generated and set on the basis of intrinsic motivations. During the initial motor-babbling, the system forms representations of sensory events. When the agent realises the possibility to re-activate those representations through its motor behaviour, it will be intrinsically motivated to improve its competence in obtaining those specific events. More precisely, the discovered events become intrinsic goals that guide both the learning and the selection of motor actions. The model is based on five components: (1) a competitive neural network, supporting the acquisition of abstract representations based on experienced changes in the
sensory input; (2) a selector that on the basis on competence-based intrinsic motivations (CB-IMs) determines the pursued goal and which motor resources will be trained to obtain that goal; (3) an echo-state neural network that controls the movements of the robot and supports the acquisition of the motor skills; (4) a predictor of the accomplishment of the pursued goal, used to measure the improvement of the system competence; (5) the generator of the CB-IM signal that biases the activity of the selector. The model is tested as the controller of a simulated simple planar robot composed of two kinematic 3DoF arms exploring own body in a 2D environment. Sensory information from self-touching is used by the system to form goals and guide skill learning. Results are presented, together with their possible implications for ongoing empirical experiments with human babies. Moreover, the model will be discussed in relation to possible applications to design new open-ended learning robotic architectures able to act in unstructured environments.
 
