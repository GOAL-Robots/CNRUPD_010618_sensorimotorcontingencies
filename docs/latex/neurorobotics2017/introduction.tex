The first ``object'' that newborn children start to play with is their own body. This activity, that starts even in the fetus and continues for many years after birth \citep{Bremner2008}, determines the formation of a ``body schema'' (cit.), a sensorimotor map and a repertoire of actions that constitute the core of future cognitive and motor development. [TODO, Kevin]

If we observe children in their first months, they seem to alternate phases of rest to phases of random activity of the limbs that gradually became more controlled and targeted during their development. Differently from this empirical evidence, in this work we propose a computational model incorporating the hypothesis that early knowledge in children is not acquired through random motor-babbling, but guided by self-generated goals, autonomously set on the basis of intrinsic motivations (IMs).

The concept of IMs was introduced in animal psychology during the 1950s and then extended in human psychology \citep{Berlyne1950,White1959,Berlyne1960,deci1985,Ryan2000}  to describe a set of motivations that were incompatible with the Hullian theory of drives \citep{Hull1943} where motivations were strictly connected to the satiation of primary needs. Different experiments \citep[e.g.][]{Harlow1950,Montgomery1954,Kish1955,Glow1978} showed how exploration, novel or surprising neutral stimuli and even the possibility to affect the environment are able to modify the behaviour of the agents driving the acquisition of knowledge and skills in the absence of tasks directly established by biological fitness. Further neurophysiological research \citep[e.g.][]{Chiodo1980,Horvitz2000,Redgrave2006} showed how IMs can be linked to neuromodulators activity, and in particular to dopamine. These results highlighted the role of IMs in enhancing neural plasticity and driving the learning of new skills. Following biological inspiration, IMs has been also introduced in machine learning \citep[e.g.][]{Barto2004,Schmidhuber2010} and developmental robotics \citep[e.g.][]{Oudeyer2007a,BaldassarreMirolliBook} to foster the autonomous development of artificial agents and the open-ended learning of repertoires of skills. Depending on their functions and mechanisms, different typologies of IMs have been identified \citep{Oudeyer2007b,Santucci2013a,Barto2013} and broadly into two main groups \citep{Baldassarre2014}: (1) knowledge-based IMs (KB-IMs), divided in (1a) novelty based IMs related to novel non-experienced stimuli, and (1b) prediction-based IMs, related to the violation of the agent's predictions; and (2) competence-based IMs (CB-IMs) related to action, i.e. to the agent's competence to change the world and accomplish self-defined $goals$. In their first implementations in computational research, IMs have been used to generate the learning signal for autonomous skills acquisition \citep{Oudeyer2007a,Hart2011,Mirolli2013,Kompella2015}. Recent research has started to use IMs for the autonomous generation and/or selection of $goals$ which can then drive the acquisition of skills \citep{Merrick2012,Baranes2013,Santucci2016} and the optimisation of learning processes in high-dimensional action spaces with redundant robot controllers \citep{Baranes2013,Rolf2014}.

Together with IMs, $goals$ are a crucial element for the presented model. Here, in consonance with computational and empirical perspectives \citep{Russell2003,Thill2013}, goals are intended as agent's internal representations of a world/body state or event (or of a set of them), with these properties: (a) the agent can keep the representation active even in the absence of the corresponding state or event; (b) the representation has the power to focus the behaviour of the agent towards the accomplishment of the goal and to generate a learning signal when the world state matches the goal (``goal-matching'').

Our hypothesis is that goals and IMs play an important role even in the early phases of knowledge acquisition, i.e. in the first months after birth. In particular, the infant initial motor-babbling results in the formation of sensory events representations. When the baby discovers (through further random activity) the possibility to re-activate the same representations through its behaviour, this will generate a CB-IM signal for obtaining those specific sensory events. More precisely, the discovered events become intrinsic goals that guide both the learning and the selection of motor actions. 

Given the important role of IMs in enhancing exploration, a different hypothesis on the development of early knowledge could exclude the involvement of an high-level construct such as goals.\\
TODO] Kevin, you should develop here the hypothesis that we discussed with you in the past (what we initially called the `oudeyer hypothesis').\\
If we remember some key points of it were as follows.\\
The agent is endowed with a repertoire of actions.\\
The agent is able to recognise a certain number of `interesting' outcomes.\\
The problem of its development is to learn a repertoire of action-outcome contingencies on the basis of intrinsic motivations.\\
The acquisition process works as follows.\\
The agent explores the environment with its actions.\\
When an interesting outcome is detected, the agent memorise in its repertoire the action-outcome contingency just experienced.\\
The action and the perceptual experience involved in the contingecy now receive a high motivation for exploration, so that the agent tends to produce similar actions and obtain similar contingencies, so learning other action outcome contingencies that are progressively stored in their repertoire.\\
When these action-outcome contingencies become progressively more refined the motivation for them decreases and exploration progressively shifts to other areas of the action-outcome space.

In this paper we only investigate our hypothesis, and leave the alternative one for future comparison. In particular, we implement a model (sec. \ref{sec:Description} and sec. \ref{sec:Implementation} tested as the controller of a simulated planar robot composed of two kinematic 3DoF arms exploring its own body in a 2D environment (sec. \ref{sec:Setup}). Sensory information from self-touch activity is used by the system to form goals and drive skill learning. Results of the experiment are presented (sec. \ref{sec:Results}) together with their possible implications for ongoing empirical experiments with human babies (sec \ref{sec:Predictions}). The final section of the paper (sec: \ref{sec:Discussion}) discusses relevant related literature and possible future development of the presented model.

%The model is tested as the controller of a simulated simple planar robot composed of two kinematic 3DoF arms exploring own body in a 2D environment. Sensory information from self-touching is used by the system to form goals and guide skill learning. Results are presented, together with their possible implications for ongoing empirical experiments with human babies.

